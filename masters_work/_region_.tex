\message{ !name(masters_work.tex)}
%%% Local Variables: 
%%% mode: latex
%%% TeX-master: t
%%% End: 

\documentclass[12pt,a4paper]{article}
\usepackage[english, ukrainian]{babel}
\usepackage[utf8]{inputenc}
\usepackage[T2A]{fontenc}
\usepackage[left=3cm,top=3cm,right=1.5cm,bottom=1.5cm,nohead,includefoot]{geometry}
\usepackage{setspace}
\usepackage{listings}
\usepackage{color}
\usepackage{float}
\usepackage{courier}
\usepackage{bold-extra}
\usepackage{fix-cm}
\usepackage{amsmath, amsthm, amssymb}

\setstretch{1.1}

\begin{document}

\message{ !name(masters_work.tex) !offset(982) }
            (this.attacks[(int)PawnTarget.SinglePush] >> 8) &
            emptySquares & rank5;
        this.attacks[(int)PawnTarget.DoublePush] =
            this.cases[this.index];

        return this.attacks;
    }
}
\end{lstlisting}

\subsubsection{Ходи тури}

Для тури створюються передобраховані масиви із всеможливих перестановок бітів
у однобайтовому числі (яке представляє собою горизонталь в шахах) та можливих
ходах у таких ситуаціях. Під час обрахунку можливих ходів для тури отримується
таке значення для горизонталі та транспонованої вертикалі, на якій стоїть
тура.


\message{ !name(masters_work.tex) !offset(998) }

\end{document}
